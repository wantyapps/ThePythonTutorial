\chapter{Basics}
\section*{Input/Output}
\subsection*{Output}
In Python 3, unlike C and C++, there is no need to use any I/O Library.
To create a simple ``Hello, World!'' output, we only need to write one line.
To do that, we need to create a Python file. We will name it \emph{hello.py}.
\emph{``hello''} is the file name, while \emph{``.py''} is the file extension.
\begin{lstlisting}
# hello.py
print('Hello, World!')
\end{lstlisting}
Let's look at the code we wrote.
At line one, we wrote ``print(\emph{\hyperref[subsec:strings]{string}})'', where \emph{string} is ``Hello, World!''.
The \emph{print()} function can take all kinds of variables, not just strings.
Notice that to create a comment, we use '\#' characters.
If we run the code with the command \emph{python3 hello.py}, the program will print:
\begin{lstlisting}[language=Text]
Hello, World!
\end{lstlisting}
We can also use \emph{Escape Sequences} in strings. For example:
\begin{lstlisting}
# hello.py
print("Hello,\nWorld")
\end{lstlisting}
outputs:
\begin{lstlisting}[language=Text]
Hello,
World!
\end{lstlisting}
Notice that we can use the \emph{'} or \emph{"} symbols for strings.
\subsection*{Input}
To get a user's keyboard input, we can use a function called \emph{input()}.
This function lets the user enter a string with a keyboard, waits for a \emph{newline} character \\
(also known as \emph{\textbackslash n} or \emph{ENTER}) and returns the string.
For example:
\begin{lstlisting}
# input.py
print(input("Name:"))
\end{lstlisting}
Take a look at the code above. In the code, we wrote ``input(\emph{string})'' where \emph{string} is ``Name:''.
The argument (\emph{string}) is the prompt; what the user will see before he is allowed to use his keyboard.
Let's run the code again:
\begin{lstlisting}[language=Text]
Name:
\end{lstlisting}
Notice that we can now use the keyboard to type something.
And after we press enter, the program will print what the user just typed.
\begin{lstlisting}[language=Text]
Name:Test
Test
\end{lstlisting}
When we look at the example above, we will see that the user typed ``Test''.
The program then printed ``Test''.
Notice that the \emph{input()} function DOES return the newline too.
\section*{Variables}
\subsection*{The \emph{type()} Function}
The \emph{type()} returns the type of the object passed to it.
For example, we can use the \emph{type()} function to print the type of a \hyperref[subsec:strings]{string}:
\begin{lstlisting}
x = "Hello"
print(type(x))
\end{lstlisting}
and it will return:
\begin{lstlisting}[language=Text]
<class 'str'>
\end{lstlisting}
or we can check the type of an \hyperref[subsec:integers]{integer} using the following code:
\begin{lstlisting}
x = 127
print(type(x))
\end{lstlisting}
\subsection*{Strings}
\label{subsec:strings}
The \emph{string} type allows us to store multiple characters.
There are two ways to create multiple-line strings. The first one is to use \emph{newlines}.
We can do that by adding a '\textbackslash{n}' character to the string.
Example:
\begin{lstlisting}
print("This is the first line.\nAnd this is the second.")
\end{lstlisting}
or, we can use three double-quotes to create a multiple-line string.
\begin{lstlisting}
print("""This is the first line.
And this is the second.""")
\end{lstlisting}
or:
\begin{lstlisting}
print('''This is the first line.
And this is the second.''')
\end{lstlisting}
\subsection*{Integers \& Floats}
\subsubsection*{Integers}
The integer type allows storage of integers.
For example, we can initialize an integer variable like the following snippet:
\begin{lstlisting}
x = 739
\end{lstlisting}
Let's look at the line above: We wrote ``x = 739''. \emph{x} is the name of the variable, and 739 is the value of it.
We can also check the type of \emph{x} using the \emph{type()} function in the following code:
\begin{lstlisting}
print(type(x))
\end{lstlisting}
which will print:
\begin{lstlisting}[language=Text]
<class 'int'>
\end{lstlisting}
\subsubsection*{Floats}
The float type stores decimal numbers, and can be initialized like the following:
\begin{lstlisting}
x = 547.23
\end{lstlisting}
and its type will be:
\begin{lstlisting}[language=Text]
<class 'float>
\end{lstlisting}
\section*{Lists/Arrays}
Lists (aka arrays) can store multiple values including strings, integers (and floats) and even other lists!

We can initialize a simple list with the following snippet:
\begin{lstlisting}
integerArray = [1, 4, 72, 7]
\end{lstlisting}
and we can access one of the values in the list using square brackets.
Example:
\begin{lstlisting}
print(integerArray[0])
\end{lstlisting}
will output:
\begin{lstlisting}[language=Text]
1
\end{lstlisting}
notice that when accessing a value in a list, the number \emph{1} points to the first value in the list,
\emph{2} points to the second value, and so forth.


We can also create a list with multiple types of values like the following:
\begin{lstlisting}
array = [16, "hello", 712.48]
\end{lstlisting}
and access it with:
\begin{lstlisting}
print(array[1])
\end{lstlisting}
or print it all with:
\begin{lstlisting}
print(array)
\end{lstlisting}
which will output the following text:
\begin{lstlisting}[language=Text]
[16, "hello", 712.48]
\end{lstlisting}


%%% Local Variables:
%%% mode: latex
%%% TeX-master: "../ThePythonTutorial"
%%% End:
